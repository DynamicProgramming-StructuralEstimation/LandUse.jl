

% preamble to include

\usepackage[utf8]{inputenc}
\usepackage{amsmath}
\usepackage{stmaryrd}
\usepackage{verbatim}
% \usepackage{color}
\usepackage{setspace}
\usepackage{pgfpages}
\usepackage{adjustbox}
\usepackage{booktabs}
\usepackage{dcolumn}
\usepackage{versions}
\usepackage{xifthen}
\usepackage{xcolor}
\usepackage{tcolorbox}
\usepackage{hyperref}
\usepackage{tikz}
\usepackage{subcaption}

\usetikzlibrary{calc}
% \usepackage{caption}
%\usepackage{beamerthemeshadow}
%\usetheme{Pittsburgh}
\usepackage{graphicx}
\usepackage{booktabs}
\usepackage{amssymb, amsfonts,amsmath}
\setbeamertemplate{footline}[frame number]
 %\setlength{\slidewidth}{9.2in}
\def\one{\mathbf{1}}
\def\l{\ell}
\def\wu{\overline{w}}
\def\wl{\underline{w}}
\def\hc{\hat{c}}
\def\hy{\hat{y}}
\def\la{\lambda}
\def\k{\kappa}
\def\g{\gamma}
\def\a{\alpha}
\def\d{\delta}
\def\Jb{\bar{J}}
\def\Jlb{\underline{J}}

\newcommand{\bm}[1]{\mbox{\boldmath$#1$}}

% red expressions
\newcommand{\re}[1]{\textcolor[rgb]{0.8,0,0}{#1}}
% blue expressions
\newcommand{\bl}[1]{\textcolor[rgb]{0,0,0.7}{#1}}
% green expressions
\newcommand{\greenie}[1]{\textcolor[rgb]{0,0.71,0.35}{#1}}
% grey expressions
\newcommand{\gr}[1]{\textcolor[rgb]{0.5,0.5,0.5}{#1}}

% itemize
\newcommand{\bi}{\begin{itemize}}
\newcommand{\ei}{\end{itemize}}

% equation display (no number)
\newcommand{\beq}{\begin{eqnarray*}}
\newcommand{\eeq}{\end{eqnarray*}}
% equation display (numbered)
\newcommand{\beqn}{\begin{eqnarray}}
\newcommand{\eeqn}{\end{eqnarray}}
\newcommand{\ti}{\frametitle}
\newcommand{\ed}{\end{document}}

\newcommand{\beginbackup}{
   \newcounter{framenumbervorappendix}
   \setcounter{framenumbervorappendix}{\value{framenumber}}
}
\newcommand{\backupend}{
   \addtocounter{framenumbervorappendix}{-\value{framenumber}}
   \addtocounter{framenumber}{\value{framenumbervorappendix}}
}

\newcommand{\db}{$HOME/Dropbox/research/LandUse}
\newcommand{\screenshots}{\db/data/manual-measurement/present-screenshots}
\newcommand{\dataplots}{\db/output/data/plots}
\newcommand{\modelplots}{\db/output/model/plots}
\newcommand{\imgs}{\db/slides/imgs}

% new commands

% define long or short presentation
%
% For now: 2 versions, 30 mins and 75 mins
%
% baseline version is the shortest presentation
% longer presentations just add slides to that short presentation 
% place additional material inside 
% \begin{v75mins}
% material to be added 
% \end{v75mins}
% environment

\newcommand{\plength}{0}  % set presentation length. set to > 0 for long presentation. can have three different lengths...
% plengths:
% 0: Baseline, 30mins
% 1: 60 mins
% 2: 75 mins
\ifthenelse{\plength > 0}{
%     \includeversion{v60mins}
%     \excludeversion{v75mins}
% }{
    % \includeversion{v60mins}
    \includeversion{v75mins}
}{
    \excludeversion{v75mins}
}


% colored text
\newcommand{\dred}{\textcolor[rgb]{0.65,0,0}}
\newcommand{\red}[1]{{\color{red}{#1}}}
\newcommand{\darkred}[1]{\dred{#1}}
\newcommand{\blue}[1]{{\color{blue}{#1}}}
\newcommand{\bred}[1]{\textbf{\dred{#1}}}
\newcommand{\bblue}[1]{\textbf{\color{blue}{#1}}}

% boxes around text
\newtcbox{\cyanbox}{on line, arc=1pt,left=1pt,right=1pt,top=1pt,bottom=1pt, colback=cyan!35!white, colframe=cyan!75!black,boxrule=1pt}
\newtcbox{\bluebox}{on line, arc=1pt,left=1pt,right=1pt,top=1pt,bottom=1pt, colback=blue!15!white, colframe=blue!80!black,boxrule=1pt}
\newtcbox{\redbox}{on line, arc=1pt,left=1pt,right=1pt,top=1pt,bottom=1pt, colback=red!5!white, colframe=red!75!black,boxrule=1pt}
\newtcbox{\greybox}{on line, arc=1pt,left=0.2pt,right=0.2pt,top=0.2pt,bottom=0.2pt, colback=black!5!white, colframe=white!75!black,boxrule=0.5pt}
\newcommand{\link}[2]{\greybox{\hyperlink{#1}{\texttt{#2}}}}
\newcommand{\slink}[2]{\greybox{\hyperlink{#1}{{\small\texttt{#2}}}}}

\newcommand{\separator}[1]{
\begin{frame}[plain]{}

\vspace{1cm}

\begin{center}
\textbf{\blue{\LARGE{}#1}}{\LARGE\par}
\par\end{center}

\end{frame}
}

% new counter
% \newcounter{saveenumi}
% \newcommand{\seti}{\setcounter{saveenumi}{\value{enumi}}}
% \newcommand{\conti}{\setcounter{enumi}{\value{saveenumi}}}

% new variable linewidth environments
% first for standard slides
\newenvironment{smalli}
{ \begin{itemize}
    \setlength{\itemsep}{1pt}
    \setlength{\parskip}{1pt}
    \setlength{\parsep}{1pt}     }
{ \end{itemize}                  } 
\newenvironment{widei}
{ \begin{itemize}
    \setlength{\itemsep}{10pt}
    \setlength{\parskip}{10pt}
    \setlength{\parsep}{10pt}     }
{ \end{itemize}                  } 
\newenvironment{smalle}
{ \begin{enumerate}
    \setlength{\itemsep}{1pt}
    \setlength{\parskip}{1pt}
    \setlength{\parsep}{1pt}     }
{ \end{enumerate}                  } 
\newenvironment{widee}
{ \begin{enumerate}
    \setlength{\itemsep}{10pt}
    \setlength{\parskip}{10pt}
    \setlength{\parsep}{10pt}     }
{ \end{enumerate}                  } 
\newenvironment{mide}
{ \begin{enumerate}
    \setlength{\itemsep}{5pt}
    \setlength{\parskip}{5pt}
    \setlength{\parsep}{5pt}     }
{ \end{enumerate}                  } 
\newenvironment{midi}
{ \begin{itemize}
    \setlength{\itemsep}{5pt}
    \setlength{\parskip}{5pt}
    \setlength{\parsep}{5pt}     }
{ \end{itemize}                  } 

% input table commands from dcolumn package
\newcolumntype{d}[1]{D{.}{.}{#1}}
\newcommand\mc[1]{\multicolumn{1}{c}{#1}} % handy shortcut macro


%%%%%%%%%%%%%%%%%%%%%%%%%%%%%%%%%%%%%%%%%%%%%
% beamer template settings
% \usetheme{Warsaw}
% \usecolortheme[RGB={200,0,0}]{structure}
% \title[Fertility And Housing]{Fertility across and within cities}

% %pour imprimer, enlever \usecolortheme[RGB={200,0,0}]{structure} et changer le documentclass en haut
% % et introduire les deux lignes suivantes
% %\usecolortheme{dove}
% %\pgfpagesuselayout{4 on 1}[a4paper,landscape,border shrink=3mm]

% \usefonttheme[onlysmall]{structurebold}

% this is for 16:9 format
% \setbeamersize{text margin left=1.5cm} \setbeamersize{text margin right=2cm}
%\setbeamercovered{transparent}




